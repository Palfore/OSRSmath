\section{Definitions}
\begin{align}
c_+ &= \frac{a}{m+1} \\
c_* &= mc_+ \\
c_0 &= 1 - c_* \\
\end{align}

\section{Recursive Equation}
The probability the player does $n$ damage to their opponent is given by:
\begin{equation}\label{eq:probability_of_n_damage}
	P(X=n) = \begin{cases}
    c_0, & \text{if } n=0\\
    c_+, & \text{if } 1 \leq n \leq m\\
    0, & \text{for } n < 0 \text{ or } m < n
    \end{cases}
\end{equation}
To be explicit, this tells us that $c_0$ is the probability of hitting a zero, and $c_+$ is the probability of doing damage. In one turn, the opponent can be brought to a given health $i$, from their initial health $h$ according to the transition probability, 
\begin{align}
	\pi_{h, i} &= \begin{cases}
		P(X=h - i) & \text{if } i > 0 \\
		P(X\ge h) & \text{if } i = 0
    \end{cases}
\end{align}
as given in Nukelawe's work. The probability they are killed in $L$ turns can be given by the sum of the probabilities the opponent was brought to $i$, then killed in $L - 1$ turns:
\begin{align}
	P_{h, L} &= \sum_{i=0}^\infty \pi_{h, i} P_{i, L-1}\\
	&= \cancel{\pi_{h, 0} P_{0, L-1}} + \pi_{h, h} P_{h, L-1} + \sum_{i=1}^{h-1} \pi_{h, i} P_{i, L-1} + \sum_{i=h+1}^\infty \pi_{h, i} P_{i, L-1}\\
	&= c_0P_{h, L-1} + \sum_{i=1}^{h-1} \pi_{h, i} P_{i, L-1} + \cancel{\sum_{i=h+1}^\infty \pi_{h, i} P_{i, L-1}}\\
	P_{h, L} &= c_0P_{h, L-1} + c_+\sum_{i=\max(h-m, 1)}^{h-1} P_{i, L-1}\\
\end{align}
where in the second line, the $i=0, h$ terms are explicitly considered, and the remaining sum is split in two. In the third line, $\pi_{h, i}$ would correspond to healing (and the $n=0$ condition in Eq.~(\ref{eq:probability_of_n_damage})) and is therefore zero. In the final line, we get the lower bound by considering that 
\begin{align}
	1\leq i \leq h-1 \implies& \pi_{h, i} = c_+ \text{ if } 1\leq h - i \leq m \text{ otherwise } 0,\,\,\, \\
	\implies& 1\leq h - i \text{ and } h - i \leq m\\
	&\therefore  i\leq h - 1 \text{ and } h - m \leq i.
\end{align}
Since the first condition is already met, we have that $i \geq h - m$, but $i$ also cannot be below 1, hence $i\geq \max(h - m, 1)$. 



% \section{Probability of killing in $L$ turns}
% 	The probability of a fight ending in $L$ turns is given by the recursive equation:
% 	\begin{align}\label{eq:recursion}
% 		P_h(L) &= c_0P_h(L - 1) + c_+ \sum_{i=\max(h - m, 1)}^{h-1} P_i(L-1),\,\,\, L \ge 2\\
% 		P_h(1) &= c_+ \max(m - h + 1, 0)
% 	\end{align}
% 	The origin of this equation is omitted until I have time to fill it in. This is a 2-dimensional coupled recursive equation.
% 	The coupling involved in the sum makes this particularly difficult. Figure~\ref{fig:grid} shows the indicies included in the sum.
% 	\begin{figure}
% 		\centering
% 		\includegraphics[width=0.5\linewidth]{grid.png}
% 		\caption{The indices used in the sum found in Eq.\ref{eq:recursion}.}
% 		\label{fig:grid}
% 	\end{figure}


% 	I was unsuccessful in directly tackling this equation 
% 	using generating functions. Instead I will manually solve several $h$ values using that technique and attempt to combine them.

% 	\subsection{$h=1$}
% 		The recursive equation to solve is:
% 		\begin{align}
% 			P_{1, L} &= c_0P_{1, L - 1},\,\,\, L \ge 2
% 		\end{align}
% 		where subscripts are used for notational convenience.

% 		Using a generating function:
% 		\begin{align}
% 			g_1(x) &= \sum_{L=1}^\infty P_{1, L} x^L \\
% 			&= P_{1,1}x + \sum_{L=2}^\infty c_0P_{1, L - 1} x^L \\
% 			&= P_{1,1}x + \sum_{L=1}^\infty c_0P_{1, L} x^{L+1} \\
% 			&= P_{1,1}x + c_0x\sum_{L=1}^\infty P_{1, L} x^{L} \\
% 			&= P_{1,1}x + c_0xg_1(x) \\
% 			\implies g_1(x) &= \frac{P_{1,1}x}{1 - c_0x} = \frac{P_{1,1}}{c_0}\frac{c_0x}{1 - c_0x}\\
% 			&= \frac{P_{1,1}}{c_0}\sum_{L=0}^\infty c_0^Lx^L\\
% 			\implies P_{1, L} &= P_{1,1}c_0^{L-1}\\
% 			P_{1, L} &= c_+\max(m, 0)c_0^{L-1}
% 		\end{align}
% 		In the second line the first term in the sum is pulled out since the recursive equation is not defined for it.

% 	\subsection{$h=2$}
% 		The recursive equation to solve is:
% 		\begin{align}
% 			P_{2, L} &= c_0P_{2, L - 1} + c_+P_{1, L-1},\,\,\, L \ge 2
% 		\end{align}
% 		Notice that the sum now contributes a term.

% 		Using a generating function:
% 		\begin{align}
% 			g_2(x) &= \sum_{L=1}^\infty P_{2, L} x^L \\
% 			&= P_{2, 1}x + \sum_{L=2}^\infty [c_0P_{2, L - 1} + c_+P_{1, L-1}] x^L \\
% 			&= P_{2, 1}x + c_0\sum_{L=2}^\infty P_{2, L - 1}x^L + c_+\sum_{L=2}^\infty P_{1, L-1} x^L \\
% 			&= P_{2, 1}x + c_0\sum_{L=1}^\infty P_{2, L}x^{L+1} + c_+\sum_{L=1}^\infty P_{1, L} x^{L+1} \\
% 			&= P_{2, 1}x + c_0xg_2(x) + c_+xg_1(x) \\
% 			\implies g_2(x) &= \frac{P_{2,1}x + c_+xg_1(x)}{1 - c_0x} \\
% 			&= \frac{P_{2,1}}{c_0}\frac{c_0x}{1 - c_0x} + \frac{c_+xg_1(x)}{1 - c_0x} \\
% 			&= \frac{P_{2,1}}{c_0}\left(\frac{c_0x}{1 - c_0x}\right)^1 + \frac{P_{1,1}c_+}{c_0^2}\left(\frac{c_0x}{1 - c_0x}\right)^2
% 		\end{align}
% 		Let's take a moment to analyze the bracketed terms since a pattern is forming. Consider the denominator:
% 		\begin{align} % https://en.wikipedia.org/wiki/Negative_binomial_distribution
% 			(1 - y)^{-r} = \sum_{k=0}^\infty {k+r-1 \choose k} y^k.
% 		\end{align}
% 		We desire the above multiplied by $y^r$:
% 		\begin{align}
% 			\left(\frac{y}{1 - y}\right)^r &= \sum_{k=0}^\infty {k+r-1 \choose k} y^{k+r} \\
% 			&= \sum_{k=r}^\infty {k-1 \choose k - r} y^{k}
% 		\end{align}
% 		Noting that $y = c_0x$, we can continue:
% 		\begin{align}
% 			g_2(x) &= \frac{P_{2,1}}{c_0} \sum_{k=1}^\infty {k-1 \choose k - 1} (c_0x)^{k} + \frac{P_{1,1}c_+}{c_0^2}\sum_{k=2}^\infty{k-1 \choose k - 2} (c_0x)^{k} \\
% 			&= P_{2,1}x + \frac{P_{2,1}}{c_0} \sum_{k=2}^\infty {k-1 \choose k - 1} (c_0x)^{k} + \frac{P_{1,1}c_+}{c_0^2}\sum_{k=2}^\infty{k-1 \choose k - 2} (c_0x)^{k} \\
% 			&= P_{2,1}x + \sum_{k=2}^\infty \left[P_{2,1}c_0^{k-1}  {k-1 \choose k - 1} + c_+P_{1,1}c_0^{k-2}{k-1 \choose k - 2}\right] x^{k}\\
% 		\end{align}
% 		Thus for $L>=2$
% 		\begin{align}
% 			\implies P_{2, L} = P_{2,1}c_0^{L-1}  {L-1 \choose L - 1} + c_+P_{1,1}c_0^{L-2}{L-1 \choose L - 2}
% 		\end{align}
% 		Now, I made a point of keeping things general, but to simplify:
% 		\begin{align}
% 			P_{2, L} &= P_{2,1}c_0^{L-1} + c_+P_{1,1}c_0^{L-2}L\\
% 			&= \left(P_{2,1} + c_+P_{1,1} \right)c_0^{L-1}
% 		\end{align}
% 		and
% 		\begin{align}
% 			P_{2, L} &= c_0^{L-1}P_{2,1} + c_+P_{1,L}\\
% 			P_{2, L} &= c_+\max(m-1, 0)c_0^{L-1} + c_+c_*c_0^{L-1}
% 		\end{align}

% 	\subsection{$h=3$}
% 		This function is now piecewise so let's split it.
% 		\subsubsection{$m=1$}
% 		The recursive equation to solve is:
% 		\begin{align}
% 			P_{3, L} &= c_0P_{3, L - 1} + c_+P_{2, L-1},\,\,\, L \ge 2
% 		\end{align}

% 		Using a generating function:
% 		\begin{align}
% 			g_3(x) &= \sum_{L=1}^\infty P_{3, L} x^L \\
% 			&= P_{3, 1}x + \sum_{L=2}^\infty [c_0P_{3, L - 1} + c_+P_{2, L-1}] x^L \\
% 			&= P_{3, 1}x + c_0\sum_{L=2}^\infty P_{3, L - 1}x^L + c_+\sum_{L=2}^\infty P_{2, L-1} x^L \\
% 			&= P_{3, 1}x + c_0\sum_{L=1}^\infty P_{3, L}x^{L+1} + c_+\sum_{L=1}^\infty P_{2, L} x^{L+1} \\
% 			&= P_{3, 1}x + c_0xg_3(x) + c_+xg_2(x) \\
% 			\implies g_3(x) &= \frac{P_{3,1}x + c_+xg_2(x)}{1 - c_0x} \\
% 		\end{align}
% 		Let's stop here since the generating function seems to have a pattern, let's see what happens if we look at $m>1$:
% 		\subsubsection{$m>1$}
% 		\begin{align}
% 			P_{3, L} &= c_0P_{3, L - 1} + c_+ P_{1, L-1} + c_+ P_{2, L-1},\,\,\, L \ge 2\\
% 		\end{align}

% 		Using a generating function:
% 		\begin{align}
% 			g_3(x) &= \sum_{L=1}^\infty P_{3, L} x^L \\
% 			&= P_{3, 1}x + \sum_{L=2}^\infty [c_0P_{3, L - 1} + c_+ P_{1, L-1} + c_+ P_{2, L-1}] x^L \\
% 			&= P_{3, 1}x + c_0\sum_{L=2}^\infty P_{3, L - 1}x^L + c_+\sum_{L=2}^\infty P_{1, L-1} x^L + c_+\sum_{L=2}^\infty P_{2, L-1} x^L \\
% 			&= P_{3, 1}x + c_0xg_3(x) + c_+xg_1(x) + c_+xg_2(x) \\
% 			\implies g_3(x) &= \frac{x}{1-c_0x}\left[P_{3,1} + c_+g_1(x) + c_+g_2(x)\right] \\
% 		\end{align}
% 		Now we can see that whatever terms show up in the recursive equation, show up in the generating function and so:
% 		\begin{align}
% 			g_h(x) &= \frac{x}{1-c_0x}\left[P_{n,1} + c_+\sum_{k=\min(m-h, 1)}^{h-1}g_k(x)\right].
% 		\end{align}
% 		Since the indicies only depend on h, we can argue that ``for a fixed $x$'' our equation has the form:
% 		\begin{align}
% 			f_h &= a\sum_{k=\delta_h}^{h-1}f_k + b,\,\,\, h \ge d + 1.
% 		\end{align}
% 		This is still non-trivial so let's try to simplify by removing the constant $b$. We will do this by finding a constant $c$ such that is cancels with b.
% 		\begin{align}
% 			(f_h + c)&= a\sum_{k=\delta_h}^{h-1}(f_k + c) + b \\
% 			f_h + c &= a\sum_{k=\delta_h}^{h-1}f_k + a\sum_{k=a}^{h-1}c + b \\
% 			f_h &= a\sum_{k=\delta_h}^{h-1}f_k + (a|I| - 1)c + b \\
% 		\end{align}
% 		Thus $b$ will vanish if:
% 		\begin{align}
% 			(a|I| - 1)c + b &=0 \\
% 			c &= \frac{b}{1-a|I|} \\
% 		\end{align}
% 		Now
% 		\begin{align}
% 			f_h &= a\sum_{k=\delta_h}^{h-1}f_k + b \implies w_h = a\sum_{k=\delta_h}^{h-1}w_k,
% 		\end{align}
% 		where $w_k = f_h + c$. Solving this recurrence requires another generating function (not to be confused with any previous):
% 		\begin{align}
% 			g(x) &= \sum_{i=\delta_h}^\infty w_i x^i \\
% 			&= w_{\delta_h}x^{\delta_h} + \sum_{i=\delta_h+1}^\infty w_i x^i \\
% 			&= w_{\delta_h}x^{\delta_h} + a\sum_{i=\delta_h+1}^\infty \sum_{k=\delta_i}^{i-1}w_k x^i \\
% 			&= w_{\delta_h}x^{\delta_h} + a\sum_{i=\delta_h}^\infty \sum_{k=\delta_i}^{i}w_k x^{i+1} \\
% 			&= w_{\delta_h}x^{\delta_h} + ax\sum_{i=\delta_h}^\infty \sum_{k=\delta_i}^{i}w_k x^{i} \\
% 			&= w_{\delta_h}x^{\delta_h} + ax\left[\sum_{i={\delta_h}}^\infty w_i x^{i} +  \sum_{i={\delta_h}}^\infty\sum_{k={\delta_h}}^{i-1}w_k x^{i} \right]\\
% 			&= w_{\delta_h}x^{\delta_h} + ax\left[g(x) + \sum_{i={\delta_h}}^\infty\sum_{k={\delta_h}}^{i}w_k x^{i+1} \right]\\
% 			% &= w_{\delta_h}x^{\delta_h} + ax\left[g(x) + \left[xg(x) + \sum_{i=d}^\infty\sum_{k=d}^{i}w_k x^{i+1} \right] \right]\\
% 			% &= w_{\delta_h}x^{\delta_h} + a\left(xg(x) + x^2g(x) + x^3g(x) + ... \right)\\
% 			% &= w_{\delta_h}x^{\delta_h} + ag(x)\left(x + x^2 + x^3 + ... \right)\\
% 			% &= w_{\delta_h}x^{\delta_h} + ag(x)\sum_{i=1}^\infty x^i\\
% 			% \implies g(x) &= \frac{w_d x^d}{1 - a\sum_{i=1}^\infty x^i}\\
% 		\end{align}
% 		part
% 		\begin{align}
% 			g(x) &= \frac{w_d x^d}{1 - a\left(\sum_{i=1}^\infty x^i +1 - 1\right)}\\
% 			g(x) &= \frac{w_d x^d}{1 - a\left(\frac{1}{1-x} - 1\right)}\\
% 			g(x) &= \frac{w_d x^d}{1 - a\frac{1}{1-x} + a}\\
% 			g(x) &= \frac{w_d x^d(1-x)}{1 - x - a + a(1-x)}\\
% 			g(x) &= \frac{w_d x^d(1-x)}{1 - (1 + a)x}\\
% 			g(x) &= \frac{w_d x^d(1-x)}{1 - (1 + a)x}\\
% 			g(x) &= w_d x^d(1-x)\sum_{i=0}^\infty(1+a)^ix^i\\
% 			g(x) &= w_d x^d\left(\sum_{i=0}^\infty(1+a)^ix^i - \sum_{i=0}^\infty(1+a)^ix^{i+1}\right)\\
% 			g(x) &= w_d x^d\left(1 + \sum_{i=1}^\infty(1+a)^ix^i - \sum_{i=1}^\infty(1+a)^{i-1}x^{i}\right)\\
% 			g(x) &= w_d x^d\left(1 + \sum_{i=1}^\infty\left[(1+a)^i - (1+a)^{i-1}\right] x^{i}\right)\\
% 			g(x) &= w_d x^d\left(1 + \sum_{i=1}^\infty (1+a)^{i-1}\left[1+a - 1\right] x^{i}\right)\\
% 			g(x) &= w_d x^d\left(1 + \sum_{i=1}^\infty\left[a(1+a)^{i-1}\right] x^{i}\right)\\
% 			g(x) &= w_d x^d + \sum_{i=1}^\infty\left[w_d a(1+a)^{i-1}\right] x^{i+d}\\
% 			g(x) &= w_d x^d + \sum_{i=d+1}^\infty \left[w_d a(1+a)^{i-d-1}\right] x^{i}\\
% 			\implies w_i(x) &= w_d a(1+a)^{i-d-1},\,\,\,i >= d + 1\\
% 		\end{align}
% 		% \begin{align}
% 		% 	g(x) &= \frac{w_d x}{1 - a\left(\sum_{i=d}^\infty x^i + \sum_{i=0}^{d-1} x^i - \sum_{i=0}^{d-1} x^i \right)}\\
% 		% 	g(x) &= \frac{w_d x}{1 - a\left(\frac{1}{1-x}- \sum_{i=0}^{d-1} x^i \right)}\\
% 		% 	g(x) &= \frac{w_d x}{1 - a\frac{1}{1-x} - a\sum_{i=0}^{d-1} x^i}\\
% 		% 	g(x) &= \frac{w_d x(1-x)}{1-x - a - a(1-x)\sum_{i=0}^{d-1} x^i }\\
% 		% 	g(x) &= \frac{w_d x(1-x)}{1-x - a - a\left(\sum_{i=0}^{d-1} x^i - \sum_{i=0}^{d-1} x^{i+1} \right)}\\
% 		% 	g(x) &= \frac{w_d x(1-x)}{1-x - a - a\left(1 -  x^d \right)}\\
% 		% 	g(x) &= w_d\frac{x(1-x)}{1-(x +  ax^d)}
% 		% \end{align}
% 		% Let $y=x+ax^d$
% 		% \begin{align}
% 		% 	g(x) &= w_dx(1-x)\frac{1}{1-y}\\
% 		% 	&= w_dx(1-x)\sum_{i=0}^\infty y^i\\
% 		% 	&= w_dx(1-x)\sum_{i=0}^\infty (x + ax^d)^i\\
% 		% 	&= w_dx(1-x)\sum_{i=0}^\infty \sum_{k=0}^i {i \choose k} a^ix^{i-k+bk}\\
% 		% \end{align}
% 		% To put this in the form of our generating function let:
% 		% \begin{align}
% 		% 	j=i+(b-1)k\\
% 		% 	k=(j-i)/(b - 1)\\
% 		% 	k=0\implies j=i\\
% 		% 	k=i\implies j=bi\\
% 		% 	a_{p, q} = {p \choose q}a^p
% 		% \end{align}
% 		% \begin{align}
% 		% 	g(x) &= w_dx(1-x)\sum_{i=0}^\infty \sum_{k=0}^i a_{i, k} x^{i-k+bk}\\
% 		% 	&= w_dx(1-x)\sum_{i=0}^\infty \sum_{j=i}^{di} a_{i, \frac{j-i}{b-1}} x^j\\
% 		% 	&= w_d\sum_{i=0}^\infty \left[ x\sum_{j=i}^{di} a_{i, \frac{j-i}{b-1}} x^j -x^2 \sum_{j=i}^{di} a_{i, \frac{j-i}{b-1}} x^j\right]\\
% 		% 	&= w_d\sum_{i=0}^\infty \left[ \sum_{j=i}^{di} a_{i, \frac{j-i}{b-1}} x^{j+1} - \sum_{j=i}^{di} a_{i, \frac{j-i}{b-1}} x^{j+2}\right]\\
% 		% 	&= w_d\sum_{i=0}^\infty \left[ \sum_{j=i+1}^{di} a_{i, \frac{j-i-1}{b-1}} x^{j} - \sum_{j=i+2}^{di} a_{i, \frac{j-i-2}{b-1}} x^{j}\right]\\
% 		% 	&= w_d\sum_{i=0}^\infty \left[ a_{i, \frac{i+1-i-1}{b-1}} x^{i+1} + \sum_{j=i+2}^{di} a_{i, \frac{j-i-1}{b-1}} x^{j} - \sum_{j=i+2}^{di} a_{i, \frac{j-i-2}{b-1}} x^{j}\right]\\
% 		% 	&= w_d\sum_{i=0}^\infty \left[ a_{i, \frac{i+1-i-1}{b-1}} x^{i+1} + \sum_{j=i+2}^{di} \left(a_{i, \frac{j-i-1}{b-1}} x^{j} - a_{i, \frac{j-i-2}{b-1}} \right)x^{j}\right]\\
% 		% 	&= w_d\sum_{i=0}^\infty  a_{i, \frac{i+1-i-1}{b-1}} x^{i+1} + w_d\sum_{i=0}^\infty\sum_{j=i+2}^{di} \left(a_{i, \frac{j-i-1}{b-1}} x^{j} - a_{i, \frac{j-i-2}{b-1}} \right)x^{j}\\
% 		% \end{align}
% 		\begin{align}
% 			&=\sum_{i=0}^\infty x^i \sum_{j=0}^{i-1-\lfloor{i/b}\rfloor} {i \choose j/(b-1)}\\
% 			\therefore w_h &= \sum_{j=0}^{h-1-\lfloor{h/b}\rfloor} {h \choose j/(b-1)}\\
% 			f_h &= \sum_{j=0}^{h-1-\lfloor{h/b}\rfloor} {h \choose j/(b-1)} + \frac{b}{1-a|I|}\\
% 			g_h(x) &= \sum_{j=0}^{h-1-\lfloor{h/b}\rfloor} {h \choose j/(b-1)} + \frac{b}{1-a|I|}\\
% 		\end{align}
	
	
% 	\subsection{Whole attempt}
% 		Let $I$ be the set of $i$'s satisfying $\max(h - m, 1) \ge i \ge h - 1$.
% 		Let $a_h=\max(m - h + 1, 0)$.
% 		The recursive equation to solve is:
% 		\begin{align}
% 			P_{h, L} &= c_0P_{h, L - 1} + c_+ \sum_{i\in I} P_{i, L-1},\,\,\, L \ge 2\\
% 		\end{align}
% 		Using a generating function:
% 		\begin{align}
% 			g(x, y) &= \sum_{h=1}^\infty\sum_{L=1}^\infty P_{h, L} y^h x^L \\
% 			&= \sum_{h=1}^\infty \left[P_{h, 1} xy^h + \sum_{L=2}^\infty P_{h, L} y^h x^L \right]\\
% 			&= \sum_{h=1}^\infty \left[c_+ a_h xy^h + \sum_{L=2}^\infty \left(c_0P_{h, L - 1} + c_+ \sum_{i\in I} P_{i, L-1}\right) y^h x^L\right]\\
% 			&= c_+x\sum_{h=1}^\infty a_h y^h + \sum_{h=1}^\infty \left[c_0\sum_{L=2}^\infty P_{h, L - 1}y^h x^L + c_+ \sum_{L=2}^\infty\sum_{i\in I} P_{i, L-1}y^h x^L \right]\\
% 			&= c_+x\sum_{h=1}^\infty a_hy^h + \sum_{h=1}^\infty \left[c_0\sum_{L=1}^\infty P_{h, L}y^h x^{L+1} + c_+ \sum_{L=1}^\infty\sum_{i\in I} P_{i, L}y^h x^{L+1} \right]\\
% 			&= c_+x\sum_{h=1}^\infty a_hy^h + c_0x\sum_{h=1}^\infty \sum_{L=1}^\infty P_{h, L}y^h x^{L} + c_+x \sum_{h=1}^\infty \sum_{L=1}^\infty\sum_{i\in I} P_{i, L}y^h x^L \\
% 			&= c_+x\sum_{h=1}^\infty a_hy^h + c_0xg(x, y) + c_+x\sum_{h=1}^\infty \sum_{L=1}^\infty\sum_{i\in I} P_{i, L}y^h x^L\\
% 		\end{align}
% 		\begin{align}
% 		\implies g(x, y) &= \frac{c_+}{c_0}\frac{c_0x}{1 - c_0x}\left[\sum_{h=1}^\infty a_hy^h +\sum_{h=1}^\infty \sum_{L=1}^\infty\sum_{i\in I} P_{i, L}y^h x^L \right]\\
% 		 &= \frac{c_+}{c_0}\left[\sum_{k=1}^\infty c_0^kx^k\sum_{h=1}^\infty a_hy^h +\sum_{k=1}^\infty c_0^kx^k \sum_{h=1}^\infty \sum_{L=1}^\infty\sum_{i\in I} P_{i, L}y^h x^L \right]\\
% 		 &= \frac{c_+}{c_0}\left[ \sum_{k=1}^\infty c_0^k\sum_{h=1}^\infty a_hy^hx^k +\sum_{k=1}^\infty \sum_{h=1}^\infty \sum_{L=1}^\infty\sum_{i\in I} c_0^kP_{i, L}y^h x^{L+k} \right]\\
% 		 &= \frac{c_+}{c_0}\left[ \sum_{k=1}^\infty c_0^k\sum_{h=1}^\infty a_hy^hx^k +\sum_{k=1}^\infty \sum_{j=k+1}^\infty \sum_{h=1}^\infty\sum_{i\in I} c_0^kP_{i, j-k}y^h x^{j} \right]\\
% 		 &= \frac{c_+}{c_0}\left[ \sum_{k=1}^\infty c_0^k\sum_{h=1}^\infty a_hy^hx^k +\sum_{n=1}^\infty \sum_{k=n+1}^\infty \left(\sum_{h=1}^\infty\sum_{i\in I} c_0^nP_{i, k-n}y^h \right) x^{k} \right]\\
% 		 &= \frac{c_+}{c_0}\left[ \sum_{k=1}^\infty c_0^k\sum_{h=1}^\infty a_hy^hx^k +\sum_{n=1}^\infty \sum_{k=n+1}^\infty a_{n,k}(y) x^{k} \right]\\
% 		 &= \frac{c_+}{c_0}\left[ \sum_{n=1}^\infty c_0^n\sum_{h=1}^\infty a_hy^hx^n + \sum_{n=1}^\infty \sum_{k=1}^{n-1} a_{k,n}(y) x^{n} \right]\\
% 		 &= \sum_{n=1}^\infty\left[ \frac{c_+}{c_0} c_0^n\sum_{h=1}^\infty a_hy^hx^n + \frac{c_+}{c_0} \sum_{k=1}^{n-1} a_{k,n}(y) \right]x^{n}\\
% 		 \implies\sum_{h=1}^\infty P_{h, L} y^h&= \frac{c_+}{c_0} c_0^k\sum_{h=1}^\infty a_h y^hx^n + \frac{c_+}{c_0} \sum_{k=1}^{n-1} a_{k,n}(y)\\
% 		 \sum_{h=1}^\infty P_{h, L} y^h &= \frac{c_+}{c_0} c_0^n\sum_{h=1}^\infty a_h y^h + \frac{c_+}{c_0} \sum_{h=1}^\infty\sum_{k=1}^{n-1} \sum_{i\in I} c_0^kP_{i, n-k}y^h\\
% 		 \sum_{h=1}^\infty P_{h, L} y^h &= \sum_{h=1}^\infty\left[ c_+c_0^{L-1} a_h + \frac{c_+}{c_0} \sum_{k=1}^{L-1} \sum_{i\in I} c_0^kP_{i, L-k} \right] y^h\\
% 		 \implies P_{h, L} &= c_+c_0^{L-1} \max(m - h + 1, 0) + c_+ \sum_{k=1}^{L-1} \sum_{i\in I} c_0^{k-1}P_{i, L-k}
% 		 \end{align}
% 		 \begin{equation}
% 		 \boxed{\therefore P_{h, L} = c_+c_0^{L-1} \max(m - h + 1, 0) + c_+ \sum_{k=1}^{L-1} \sum_{i\in I} c_0^{k-1}P_{i, L-k}}
% 		 \end{equation}


% 	\subsection{Second Whole Attempt}
\section{Solution}
	The recursive equation to solve is:
	\begin{align}
		P_{h, L} &= c_0P_{h, L - 1} + c_+ \sum_{i\in I_h} P_{i, L-1},\,\,\, L \ge 2, h \ge 1\\ % maybe 2
	\end{align}
	where $I_h$ is the set of integers satisfying $h - 1 \ge i \ge \max(h - m, 1)$. The initial conditions are given by:
	\begin{align}
		P_{h, 1} &= c_+ \max(m - h + 1, 0)\\
		P_{1, L} &= c_* c_0^{L-1}\\
		P_{1, 1} &= c_*
	\end{align}
	
	Using a generating function:
	\begin{align}
		g(x, y) &= \sum_{h=1}^\infty\sum_{L=1}^\infty P_{h, L} y^h x^L \\
		&= \sum_{h=1}^\infty\left(P_{h, 1} y^h x + \sum_{L=2}^\infty P_{h, L} y^h x^L \right)\\
		&= \sum_{h=1}^\infty P_{h, 1} y^h x + \sum_{h=1}^\infty\sum_{L=2}^\infty P_{h, L} y^h x^L\\
		&= xyP_{1, 1} + \sum_{h=2}^\infty P_{h, 1} y^h x + \sum_{L=2}^\infty\left( P_{1,L} yx^L + \sum_{h=2}^\infty P_{h, L} y^h x^L \right)\\
		&= xyP_{1, 1}  + x\sum_{h=2}^\infty P_{h, 1} y^h + \sum_{L=2}^\infty P_{1,L} yx^L + \sum_{L=2}^\infty\sum_{h=2}^\infty P_{h, L} y^h x \\
		&= xyP_{1, 1}  + x\sum_{h=2}^\infty P_{h, 1} y^h  + y\sum_{L=2}^\infty P_{1,L} x^L + \sum_{L=2}^\infty\sum_{h=2}^\infty P_{h, L} y^hx^L \\
	\end{align}
	These are the boundaries of a grid (corner $+$ top $+$ left) plus a sum over the interior. 

	\subsection{Corner} 
	Let's focus on each term at a time, starting with the corner:
	\begin{align}
		xyP_{1, 1} = xyc_*
	\end{align}
	\subsection{Top}
	For the top, lets first note that $\max(m - h + 1, 0)$ is non-zero when $m - h + 1 \ge 1 \implies m \ge h$. Then,
	\begin{align}
		x\sum_{h=2}^\infty P_{h, 1} y^h &= c_+x \sum_{h=2}^\infty\max(m - h + 1, 0)y^h\\
		&= c_+x \sum_{h=2}^m (m - h + 1)y^h\\
	\end{align}
	\subsection{Left}
	Now the left:
	\begin{align}
		y\sum_{L=2}^\infty P_{1,L} x^L &= y\sum_{L=2}^\infty c_* c_0^{L-1} x^L \\
		&= y\frac{c_*}{c_0}\sum_{L=2}^\infty (c_0x)^L \\
		&= y\frac{c_*}{c_0}\left[\sum_{L=0}^\infty (c_0x)^L - 1 - c_0x\right]\\
		&= y\frac{c_*}{c_0}\left[\frac{1}{1-c_0x} - 1 - c_0x\right]\\
		&= y\frac{c_*}{c_0}\frac{1}{1-c_0x}\left[1 - (1-c_0x) - (1-c_0x)c_0x\right]\\
		&= y\frac{c_*}{c_0}\frac{1}{1-c_0x}\left[c_0^2x^2\right]\\
		&= y\frac{c_*}{c_0}\frac{c_0^2x^2}{1-c_0x}\\
	\end{align}
	\subsection{Interior}
	Now the interior:
	\begin{align}
		\sum_{L=2}^\infty\sum_{h=2}^\infty P_{h, L} y^hx^L &= \sum_{L=2}^\infty\sum_{h=2}^\infty \left(c_0P_{h, L - 1} + c_+ \sum_{i\in I_h} P_{i, L-1}\right) y^hx^L\\
		&= \sum_{L=2}^\infty\sum_{h=2}^\infty c_0P_{h, L - 1}y^hx^L + c_+ \sum_{L=2}^\infty\sum_{h=2}^\infty\sum_{i\in I_h} P_{i, L-1}y^hx^L\\
		\mathcal{I}(x, y)&= G(x, y) + R(x, y).
	\end{align}
	Let us also tackle this individually, starting with the `g' term:
	\begin{align}
		\sum_{L=2}^\infty\sum_{h=2}^\infty c_0P_{h, L - 1}y^hx^L &= c_0\sum_{L=1}^\infty\sum_{h=2}^\infty P_{h, L}y^hx^{L+1}\\
		&= c_0x\sum_{L=1}^\infty\sum_{h=2}^\infty P_{h, L}y^hx^L\\
		&= c_0x\sum_{L=1}^\infty\left( \sum_{h=1}^\infty P_{h, L}y^hx^L -  P_{1, L}yx^L\right)\\
		&= c_0x\sum_{L=1}^\infty\sum_{h=1}^\infty P_{h, L}y^hx^L -  c_0xy\sum_{L=1}^\infty P_{1, L}x^L\\
		&= c_0xg(x, y) -  c_* c_0xy\sum_{L=1}^\infty c_0^{L-1}x^L\\
		&= c_0xg(x, y) -  c_* c_0x^2y\sum_{L=1}^\infty c_0^{L-1}x^{L-1}\\
		&= c_0xg(x, y) -  c_* c_0 x^2y \sum_{L=0}^\infty c_0^{L}x^L\\
		&= c_0xg(x, y) -  \frac{c_* c_0x^2y}{1-c_0x}\\
	\end{align}
	
	% Followed by $R$. 
	% First let's note that $I$ is a non-empty set when $h-1>=\max(m-h, 1)$, empirically this is equivalent to:
	% $$h\ge \max\left(\left\lceil \frac{m + 1}{2} \right\rceil, 2\right)\equiv M \ge 2$$
	For $R$, we will first need a term that tells us whether $h$ is in the set $I$, i.e. does $h$ satisfy $h-1>=\max(m-h, 1)$? You will actually see that we need the more general $h-1>=\max(m-h+n, 1)$. We will call this condition $\delta_{m, h}^n$ which is $1$ if satisfied and $0$ otherwise. Empirically, this can be expressed as:
	\begin{align}
		\delta_{m, h}^n = \begin{cases}
			0 & \text{ if } h = 1 \\
			0 & \text{ if } n > m \\
			1 & \text{ otherwise } \\
		\end{cases}
	\end{align}
	Since $h >=2$,
	\begin{align}
		\delta_{m, h}^n = \delta_m^n = \begin{cases}
			1 & \text{ if } n \le m\\
			0 & \text{ otherwise } \\
		\end{cases}
	\end{align}
	Now solving $R(x, y)= c_+ \sum_{L=2}^\infty\sum_{h=2}^\infty\sum_{i\in I} P_{i, L-1}y^hx^L$ gives:
	\begin{align}
		R(x, y) &= c_+ \sum_{L=2}^\infty\sum_{h=2}^\infty\sum_{i=\max(h-m, 1)}^{h-1} P_{i, L-1}y^hx^L\\
		&= c_+x \sum_{L=1}^\infty\sum_{h=2}^\infty\sum_{i=\max(h-m, 1)}^{h-1} P_{i, L}y^hx^L\\
		&= c_+xy \sum_{L=1}^\infty\sum_{h=1}^\infty\sum_{i=\max(h-m+1, 1)}^{h} P_{i, L}y^hx^L\\
		&= c_+xy \sum_{L=1}^\infty\sum_{h=1}^\infty\left(P_{h, L}\delta_m^1 + \sum_{i=\max(h-m+1, 1)}^{h-1} P_{i, L}\right)y^hx^L\\
		&= c_+xy \sum_{L=1}^\infty\sum_{h=1}^\infty P_{h, L}x^L\delta_m^1 + c_+xy \sum_{L=1}^\infty\sum_{h=1}^\infty\sum_{i=\max(h-m+1, 1)}^{h-1} P_{i, L}y^hx^L\\
	\end{align}
	Notice that the $h=1$ term in the second set of sums yields 0.
	\begin{align}
		&= c_+xy g(x, y)\delta_h^1 + c_+xy \sum_{L=1}^\infty\sum_{h=2}^\infty\sum_{i=\max(h-m+1, 1)}^{h-1} P_{i, L}y^hx^L\\
		&= c_+xy g(x, y)\delta_h^1 + c_+xy^2 \sum_{L=1}^\infty\sum_{h=1}^\infty\sum_{i=\max(h-m+2, 1)}^{h} P_{i, L}y^hx^L\\
		&= c_+xy g(x, y)\delta_h^1 + c_+xy^2 \sum_{L=1}^\infty\sum_{h=1}^\infty\left(\sum_{i=\max(h-m+2, 1)}^{h} P_{i, L}\right)y^hx^L\\
		&= c_+xy g(x, y)\delta_h^1 + c_+xy^2 \sum_{L=1}^\infty\sum_{h=1}^\infty\left(P_{i, L}\delta_m^2 + \sum_{i=\max(h-m+2, 1)}^{h} P_{i, L}\right)y^hx^L\\
		&= c_+xy g(x, y)\delta_h^1 + c_+xy^2 \sum_{L=1}^\infty\sum_{h=1}^\infty P_{i, L}\delta_m^2 + c_+xy^2 \sum_{L=1}^\infty\sum_{h=1}^\infty\sum_{i=\max(h-m+2, 1)}^{h} P_{i, L}y^hx^L\\
		&= c_+xy g(x, y)\delta_h^1 + c_+xy^2 g(x, y)\delta_m^2 + c_+xy^2 \sum_{L=1}^\infty\sum_{h=1}^\infty\sum_{i=\max(h-m+2, 1)}^{h} P_{i, L}y^hx^L\\
	\end{align}
	These series continues until the `engine' producing terms has no more. The number of terms in this series is given by the maximum $n$ that is non-zero:
	\begin{align}
		\arg \max_n \delta_m^n = m,
	\end{align}
	and so,
	\begin{align}
		R(x, y) &= c_+ x g(x, y)\sum_{i=1}^m y^i
	\end{align}
	% Followed by the recursive term (you'll see where it gets the name):
	% \begin{align}
	% 	c_+ \sum_{L=2}^\infty\sum_{h=2}^\infty\sum_{i\in I} P_{i, L-1}y^hx^L &= c_+ \sum_{L=1}^\infty\sum_{h=2}^\infty\sum_{i\in I_h} P_{i, L}y^hx^{L+1}\\
	% 	&= c_+ \sum_{h=2}^\infty\sum_{L=1}^\infty\sum_{i\in I_h} P_{i, L}y^hx^{L+1}\\
	% 	&= c_+ x\sum_{L=1}^\infty\sum_{h=2}^\infty\sum_{i\in I_h} P_{i, L}y^hx^L\\
	% 	&= c_+ x\sum_{L=1}^\infty x^L\left[\sum_{h=2}^\infty \sum_{i=\max(m-h, 1)}^{h-1} P_{i, L}y^h\right]\\
	% 	&= c_+ x\sum_{L=1}^\infty x^L\left[y\sum_{h=1}^\infty \sum_{i=\max(m-h-1, 1)}^{h} P_{i, L}y^h\right]\\
	% 	&= c_+ x\sum_{L=1}^\infty x^L\left[y\sum_{h=1}^\infty \left(P_{h, L}y^h + \sum_{i=\max(m-h-1, 1)}^{h-1} P_{i, L}y^h \right)\right]\\
	% 	&= c_+ x\sum_{L=1}^\infty x^L\left[y\sum_{h=1}^\infty P_{h, L}y^h + y\sum_{h=1}^\infty\sum_{i=\max(m-h-1, 1)}^{h-1} P_{i, L}y^h \right]\\
	% 	&= c_+ xy\sum_{L=1}^\infty \sum_{h=1}^\infty P_{h, L}y^hx^L + c_+ xy\sum_{L=1}^\infty x^L\sum_{h=1}^\infty\sum_{i=\max(m-h-1, 1)}^{h-1} P_{i, L}y^h \\
	% 	&= c_+ xyg(x, y) + c_+ xy\sum_{L=1}^\infty \sum_{h=1}^\infty\sum_{i=\max(m-h-1, 1)}^{h-1} P_{i, L}y^hx^L \\
	% 	&= c_+ xg(x, y) \left(y + y^2 + ...\right) \\
	% 	&= c_+ xg(x, y) \frac{1}{1-y} \\
	% \end{align}
	Now let's combine everything:
	\begin{align}
		g(x, y) &= xyc_* + c_+x \sum_{h=2}^m (m - h + 1)y^h + \cancel{y\frac{c_*}{c_0}\frac{c_0^2x^2}{1-c_0x}} + c_0xg(x, y) - \cancel{\frac{c_* c_0x^2y}{1-c_0x}} +  c_+ x g(x, y)\sum_{i=1}^m y^i\\
		&= c_+x \sum_{h=1}^m (m - h + 1)y^h + c_0xg(x, y) + c_+ x g(x, y)\sum_{i=1}^m y^i
	\end{align}
	Isolating for $g(x, y)$:
	\begin{align}
		g(x, y) - c_0xg(x, y) - c_+ x g(x, y)\sum_{i=1}^m y^i &= c_+x \sum_{h=1}^m (m - h + 1)y^h\\
		\left(1 - c_0x - c_+ x\sum_{i=1}^m y^i\right)g(x, y) &= c_+x \sum_{h=1}^m (m - h + 1)y^h\\
		g(x, y) &= x\frac{ c_+ \sum_{h=1}^m (m - h + 1)y^h}{1 - \left(c_0 + c_+ \sum_{i=1}^m y^i\right)x}\\
		g(x, y) = \sum_{h=1}^\infty\sum_{L=1}^\infty P_{h, L} y^h x^L &= x\frac{c_+ \sum_{h=1}^m (m - h + 1)y^h}{1 - \left(c_0 + c_+ \sum_{i=1}^m y^i\right)x}
	\end{align}
	To simplify, let's define:
	\begin{align}
		T(y) &= c_+ \sum_{h=1}^m (m - h + 1)y^h\\
		B(y) &= c_0 + c_+ \sum_{i=1}^m y^i\\
	\end{align}
	Now $g(x, y)$ can be written as,
	\begin{align}
		g(x, y) &= T(y)\frac{x}{1- B(y)x}.
	\end{align}
	\newpage
	\subsection{Obtaining Power Series}
	%%%%%%%%%%%%%%%%%%%%%%%%%%%%%%%%%%%%%%%%%%%%%%%%%%%%%%%%%%%%%%%%
	To find the series representation of this, we will require the following identities:
	\begin{equation}\label{eq:binomial}
		(x + y)^n = \sum_{k=0}^n {n \choose k} x^{n-k}y^k
	\end{equation}
	\begin{equation}\label{eq:sub_binomial}
		(x - y)^n = \sum_{k=0}^n {n \choose k} (-1)^k x^{n-k}y^k
	\end{equation}
	\begin{equation}\label{eq:binomial_special}
		\frac{1}{(1-z)^\beta} = \sum_{k=0}^\infty {k + \beta -1 \choose k} z^k
	\end{equation}
	\begin{equation}\label{eq:geometric}
		\sum_{k=0}^n r^k = \frac{1 - r^{n+1}}{1 - r}
	\end{equation}
	%%%%%%%%%%%%%%%%%%%%%%%%%%%%%%%%%%%%%%%%%%%%%%%%%%%%%%%%%%%%%%%%

	Let us start with:
	\begin{align}
		\frac{1}{1- B(y)x} &= \sum_{k=0}^\infty B^k(y)x^k \\
	\end{align}
	We need to workout $B^k(y)$:
	\begin{align}
	B^k(y) &= \left( c_0 + c_+ \sum_{j=1}^m y^j\right)^k  \\
		&= \sum_{i=0}^k {k \choose i} c_0^{k-i} c_+^i \left(\sum_{j=1}^m y^j\right)^i  \\
		&= c_0^k \sum_{i=0}^k {k \choose i}  \left(\frac{c_+}{c_0}\right)^i \left( \sum_{j=1}^m y^j\right)^i  \\
	\end{align}
	Now, we would like to handle the last term,
	\begin{align}
	\left( \sum_{j=1}^m y^j \right)^i &=  \left( \sum_{j=0}^m y^j - 1 \right)^i \\
		&= \left( \frac{1 - y^{m+1}}{1 - y} - 1 \right)^i \\
		&= \left( \frac{1 - y^{m+1} - 1 + y}{1 - y} \right)^i\\
		&= \frac{(y - y^{m+1})^i}{(1 - y)^i} \\
		&= (y - y^{m+1})^i \cdot \sum_{j=0}^\infty {j+i-1\choose j}y^j \\
		&= (y - y^{m+1})^i \cdot \sum_{j=0}^\infty {j+i-1\choose j}y^j \\
		&= \sum_{l=0}^i {i \choose l} (-1)^l y^{i+lm} \cdot \sum_{j=0}^\infty {j+i-1\choose j}y^j \\
		&= \sum_{j=0}^\infty\sum_{l=0}^i  (-1)^l  {i \choose l} {j+i-1\choose j}y^{i+j+lm} \\
	\end{align}
	where in the first line, Eq.~(\ref{eq:binomial_special}) was used. In the second line, Eq.~(\ref{eq:sub_binomial}) was used. We also note that the expansion requires $i>1$.
	We can simplify this by defining,
	\begin{align}
		a_{l, j}^i \equiv (-1)^l  {i \choose l} {j+i-1\choose j}
	\end{align}
	leaving us with
	\begin{equation}
		\left( \sum_{j=1}^m y^j \right)^i = y^i\sum_{j=0}^\infty\sum_{l=0}^i  a_{l, j}^i y^{j+lm} \\
	\end{equation}

	This needs to be cast as a regular power series. For this, we turn to a visual proof [Omitted], which yields:
	\begin{equation}
		\sum_{j=0}^\infty \sum_{l=0}^i a_{l, j}^i y^{j + ml} = \sum_{j=0}^\infty \left( \sum_{l=0}^{\min(\lfloor j / m \rfloor, i)} a_{l, j-ml}^i\right)y^j.
	\end{equation}
	If we define,
	\begin{align}
		A_{j, i} \equiv \sum_{l=0}^{\min(\lfloor j / m \rfloor, i)} a_{l, j-ml}^i,
	\end{align}
	then,
	\begin{equation}
		\sum_{j=0}^\infty \sum_{l=0}^i a_{l, j}^i y^{j+ml} = \sum_{j=0}^\infty A_{j, i}y^j.
	\end{equation}



	And so,
	\begin{equation}
		\left( \sum_{j=1}^m y^j \right)^i = y^i\sum_{j=0}^\infty A_{j, i}y^j. \\
	\end{equation}
	Expanding fully gives,
	\begin{equation}
		\left( \sum_{j=1}^m y^j \right)^i = \sum_{j=0}^\infty \left[ \sum_{l=0}^{\min(\lfloor j/m\rfloor, i)} (-1)^l {i \choose l} {j - ml + i - 1 \choose j - ml} \right]y^{i+j}. \\
	\end{equation}
	Returning to $B^k(y)$:
	\begin{align}
		B^k(y) &= c_0^k \sum_{i=0}^k {k \choose i}  \left(\frac{c_+}{c_0}\right)^i \left( \sum_{j=1}^m y^j\right)^i\\
		&= c_0^k  \left[ 1 + \sum_{i=1}^k {k \choose i}  \left(\frac{c_+}{c_0}\right)^i  \sum_{j=0}^\infty A_{j, i}y^{i+j}\right]\\
		&= c_0^k  \left[ 1 +  \sum_{j=0}^\infty\sum_{i=1}^k {k \choose i}  \left(\frac{c_+}{c_0}\right)^i  A_{j, i}y^{i+j}\right]\\
		% &= c_0^k  \left[ 1 +  \sum_{j=0}^\infty\left(\sum_{n=j+1}^{j+k} {k \choose n-j}  \left(\frac{c_+}{c_0}\right)^{n-j}  A_{j, n-j}\right)y^n\right]\\
	\end{align}

	To handle this, we'll define,
	\begin{align}
		D_{i, j}^k = {k \choose i}  \left(\frac{c_+}{c_0}\right)^i  A_{j, i},
	\end{align}
	making,
	\begin{align}
		B^k(y) = c_0^k  \left[ 1 +  \sum_{j=0}^\infty\sum_{i=1}^k D_{i, j}^k y^{i+j}\right],
	\end{align}
	and make use of the identity [visual proof is also omitted]:
	\begin{align}
		\sum_{j=0}^\infty \sum_{i=1}^k a_{i, j} y^{i+j} = \sum_{j=1}^\infty\left( \sum_{i=1}^{\min(j, k)}a_{i, j-i} \right) y^j.
	\end{align}
	This leaves us with,
	\begin{align}
		B^k(y) = c_0^k  \left[ 1 +  \sum_{j=1}^\infty\left( \sum_{i=1}^{\min(j, k)}D_{i, j-i}^k \right) y^j\right].
	\end{align}
	We further simplify:
	\begin{align}
		F_{j, k} \equiv \sum_{i=1}^{\min(j, k)}D_{i, j-i}^k,
	\end{align}
	leaving us finally with:
	\begin{align}
		B^k(y) = c_0^k  \left[ 1 +  \sum_{j=1}^\infty F_{j, k} y^j\right].
	\end{align}

	So our original formula becomes,
	\begin{align}
		\frac{1}{1-B(y)x} &= \sum_{k=0}^\infty c_0^k  \left[ 1 +  \sum_{j=1}^\infty F_{j, k} y^j\right] x^k \\
		\implies\frac{x}{1-B(y)x} &= \sum_{k=1}^\infty c_0^{k-1}  \left[ 1 +  \sum_{j=1}^\infty F_{j, k-1} y^j\right] x^{k} \\
		&= \sum_{k=1}^\infty c_0^{k-1}x^{k} +  \sum_{k=1}^\infty \sum_{j=1}^\infty c_0^{k-1} F_{j, k-1} y^jx^{k} \\
	\end{align}

	Multiplying by $T(y)$ gives back our generating function:
	\begin{align}
		g(x, y) = \frac{T(y)x}{1-B(y)x} &= T(y)\sum_{k=1}^\infty c_0^{k-1}x^{k} +  T(y)\sum_{k=1}^\infty \sum_{j=1}^\infty c_0^{k-1} F_{j, k-1} y^jx^{k} \\
		&= c_+ \sum_{h=1}^m (m - h + 1)y^h\sum_{k=1}^\infty c_0^{k-1}x^{k} +  T(y)\sum_{k=1}^\infty \sum_{j=1}^\infty c_0^{k-1} F_{j, k-1} y^jx^{k} \\
		&= \sum_{h=1}^m \sum_{k=1}^\infty \left[ c_+ c_0^{k-1}(m - h + 1) \right]y^hx^{k} +  T(y)\sum_{k=1}^\infty \sum_{j=1}^\infty c_0^{k-1} F_{j, k-1} y^jx^{k} \\
	\end{align}
	Focusing on the second series:
	\begin{align}
		T(y)\sum_{k=1}^\infty \sum_{j=1}^\infty c_0^{k-1} F_{j, k-1} y^jx^{k} &= c_+ \sum_{h=1}^m (m - h + 1)y^h\sum_{k=1}^\infty \sum_{j=1}^\infty c_0^{k-1} F_{j, k-1} y^jx^{k}\\
		&=  \sum_{h=1}^m \sum_{k=1}^\infty \sum_{j=1}^\infty c_+c_0^{k-1} (m - h + 1)F_{j, k-1} y^{j+h}x^{k}\\
		&=  \sum_{k=1}^\infty  \left[ \sum_{j=1}^\infty \sum_{h=1}^m c_+c_0^{k-1} (m - h + 1)F_{j, k-1} y^{j+h}\right]x^{k}\\
	\end{align}
	Defining
	\begin{align}
		H_{j, h}^k &\equiv   c_+c_0^{k-1} (m - h + 1)F_{j, k-1},
	\end{align}
	makes,
	\begin{align}
		T(y)\sum_{k=1}^\infty \sum_{j=1}^\infty c_0^{k-1} F_{j, k-1} y^jx^{k} &=  \sum_{k=1}^\infty  \left[ \sum_{j=1}^\infty \sum_{h=1}^m H_{j, h}^k y^{j+h}\right]x^{k}\\
		&=  \sum_{k=1}^\infty  \left[ \sum_{j=1}^\infty \sum_{h=1}^{\min(j, m)} H_{j-h, h}^k y^j\right]x^{k}\\
		&=  \sum_{k=1}^\infty \sum_{j=1}^\infty \left(\sum_{h=1}^{\min(j, m)} H_{j-h, h}^k \right) y^jx^{k}\\
	\end{align}
	So,
	\begin{align}
		g(x, y) &= \sum_{L=1}^\infty \sum_{h=1}^m  c_+ c_0^{L-1}(m - h + 1) y^hx^L + \sum_{L=1}^\infty \sum_{h=1}^\infty \left(\sum_{i=1}^{\min(h, m)} H_{h-i, i}^L \right) y^hx^{L}\\
		&= \sum_{L=1}^\infty \sum_{h=1}^\infty \theta(m-h) c_+ c_0^{L-1}(m - h + 1) y^hx^L + \sum_{L=1}^\infty \sum_{h=1}^\infty \left(\sum_{i=1}^{\min(h, m)} H_{h-i, i}^L \right) y^hx^{L}\\
		&= \sum_{L=1}^\infty \sum_{h=1}^\infty \left[ \theta(m-h)c_+ c_0^{L-1}(m - h + 1) + \sum_{i=1}^{\min(h, m)} H_{h-i, i}^L\right] y^hx^{L}\\
	\end{align}
	where $\theta(x)$ is the Heaviside function. And at last, we obtain:
	\begin{equation}
		P_{h, L} = \theta(m-h)c_+ c_0^{L-1}(m - h + 1) + \sum_{i=1}^{\min(h, m)} H_{h-i, i}^L
	\end{equation}
	\begin{equation}
		\boxed{\therefore P_{h, L} = c_+ c_0^{L-1}\max(m - h + 1, 0) + \sum_{i=1}^{\min(h, m)} H_{h-i, i}^L }
	\end{equation}
	%P_{h, L} =\frac{P_{1, L}P_{h, 1}}{P_{1,1}} + \sum_{i=1}^{\min(h, m)} H_{h-i, i}^L\\
	We can unravel $H$ using:
	\begin{align}
		H_{j, h}^k &\equiv   c_+c_0^{k-1} (m - h + 1)F_{j, k-1},\\
		F_{j, k} &\equiv \sum_{i=1}^{\min(j, k)}D_{i, j-i}^k,\\
		D_{i, j}^k &= {k \choose i}  \left(\frac{c_+}{c_0}\right)^i  A_{j, i},\\
		A_{j, i} &\equiv \sum_{l=0}^{\min(\lfloor j / m \rfloor, i)} a_{l, j-ml}^i,\\
		a_{l, j}^i &\equiv (-1)^l  {i \choose l} {j+i-1\choose j}\\
	\end{align}
	to get:
	\begin{align}
		H_{h-i, i}^L &= c_+c_0^{L-1} (m - i + 1)F_{h-i, L-1}\\
		&= C_ic_0^{L-1} \sum_{p=1}^{\min_{h-i}^{L-1}} D_{p, h-i-p}^{L-1}\\
		&= C_ic_0^{L-1} \sum_{p=1}^{\min_{h-i}^{L-1}} {L-1 \choose p}  \left(\frac{c_+}{c_0}\right)^p  A_{h-i-p, p}\\
		&= C_ic_0^{L-1} \sum_{p=1}^{\min_{h-i}^{L-1}} {L-1 \choose p}  \left(\frac{c_+}{c_0}\right)^p  \sum_{l=0}^{\min_{\lfloor (h-i-p) / m \rfloor}^{p}} a_{l, h-i-p-ml}^p\\
		&= C_ic_0^{L-1} \sum_{p=1}^{\min_{h-i}^{L-1}} {L-1 \choose p}  \left(\frac{c_+}{c_0}\right)^p  \sum_{l=0}^{\min_{\lfloor (h-i-p) / m \rfloor}^{p}} (-1)^l  {p \choose l} {h-i-ml-1\choose h-i-ml-p}\\
		&= C_i c_0^{L-1}\sum_{p=1}^{\min_{h-i}^{L-1}} {L-1 \choose p}  G(h, i, p)\\
		%F_{h-i, k-1} &\equiv \sum_{p=1}^{\min(h-i, k-1)}D_{p, h-i-p}^{k-1},\\
		%D_{p, h-i-p}^{k-1} &= {k-1 \choose p}  \left(\frac{c_+}{c_0}\right)^p  A_{h-i-p, p},\\
		%A_{h-i-p, p} &\equiv \sum_{l=0}^{\min_{\lfloor (h-i-p) / m \rfloor}^{p}} a_{l, h-i-p-ml}^p,\\
		% a_{l, h-i-p-ml}^p &\equiv (-1)^l  {p \choose l} {h-i-ml-1\choose h-i-p-ml}\\
	\end{align}
	where $C_i = c_+(m - i + 1)$ and 
	\begin{align}
		G(h, i, p) &\equiv \left(\frac{c_+}{c_0}\right)^p  \sum_{l=0}^{\min_{\lfloor (h-i-p) / m \rfloor}^{p}} (-1)^l  {p \choose l} {h-i-ml-1\choose h-i-ml-p}.
	\end{align}
	% Expanding $H$ and identifying the boundary conditions yields,
	% \begin{equation}
	% 	\boxed{P_{h, L} = \frac{P_{1, L}P_{h, 1}}{P_{1,1}} + c_+c_0^{L-1} \sum_{i=1}^{\min^h_m} \sum_{l=1}^{\min^{h - i}_{L - 1}} \sum_{q=0}^{\min^{\lfloor\frac{h - i - l}{m}\rfloor}_l} (-1)^q(m - i + 1) \left( \frac{c_+}{c_0} \right)^l {L- 1 \choose l}  {l \choose q} {h - i - mq -1 \choose h - i - mq - l},}
	% \end{equation}
	% where the notation $\min(a, b) \equiv \min^a_b \equiv \min^b_a$ was used to shorten the expression.


	\section{Summing over $L$}
		Often, the following quantity needs to be evaluated for further applications, like the probability of winning a fight:
		\begin{equation}
			\sum_{L=b}^\infty P_{h, L} = P_{h, 1}\sum_{L=b}^\infty c_0^{L-1}  + \sum_{i=1}^{\min(h, m)} \sum_{L=b}^\infty H_{h, i}^L,
		\end{equation}
		

		Tackling the first term, we get: % https://en.wikipedia.org/wiki/Geometric_progression, related formulas
		\begin{align}
			\sum_{L=b}^\infty c_0^{L-1} &= \sum_{L=b-1}^\infty c_0^{L} \\
			&= \frac{c_0^{b-1}}{1 - c_0} \\
		\end{align}

		% \newpage
		For the second term, we need:
		\begin{align}
			\sum_{i=a}^\infty {i \choose k} x^i &= \sum_{i=1}^\infty {i \choose k} x^i - \sum_{i=1}^{a-1} {i \choose k} x^i,\,\,\,1\leq k \leq a\\
			&= \frac{x^k}{(1-x)^{k+1}} - \sum_{i=1}^{a-1} {i \choose k} x^i\\
		\end{align}

		Focusing on the second term,
		\begin{align}
			\sum_{L=b}^\infty H_{h, i}^L &= C_i \sum_{L=b}^\infty  c_0^{L-1} \sum_{p=1}^{\min_{h-i}^{L-1}} {L-1 \choose p}  G(h, i, p)\\
			&= C_i \sum_{L=b}^\infty  \left[ \sum_{p=1}^{\min_{h-i}^{L-1}} c_0^{L-1}{L-1 \choose p}  G(h, i, p) \right]\\
			&= C_i \sum_{L=b}^{h-i} \sum_{p=1}^{L-1} c_0^{L-1}{L-1 \choose p}  G(h, i, p) + C_i \sum_{L=\max^b_{h-i+1}}^{\infty} \sum_{p=1}^{h-i} c_0^{L-1}{L-1 \choose p}  G(h, i, p)\\
			&= C_i \sum_{L=b}^{h-i} \sum_{p=1}^{L-1} c_0^{L-1}{L-1 \choose p}  G(h, i, p) + C_i  \sum_{p=1}^{h-i} \left[\sum_{L=\max^b_{h-i+1}}^{\infty} c_0^{L-1}{L-1 \choose p}  \right]G(h, i, p)\\
			&= C_i \sum_{L=b}^{h-i} \sum_{p=1}^{L-1} c_0^{L-1}{L-1 \choose p}  G(h, i, p) + C_i  \sum_{p=1}^{h-i} \left[\sum_{L=\max^{b-1}_{h-i}}^{\infty} c_0^{L}{L \choose p}  \right]G(h, i, p)\\
			&= C_i \sum_{L=b}^{h-i} \sum_{p=1}^{L-1} c_0^{L-1}{L-1 \choose p}  G(h, i, p) + C_i  \sum_{p=1}^{h-i} \left[\sum_{L=0}^{\infty} c_0^{L}{L \choose p} - \sum_{L=0}^{\max^{b - 1}_{h - i} - 1} c_0^{L}{L \choose p}  \right]G(h, i, p)\\
			&= C_i \sum_{L=b}^{h-i} \sum_{p=1}^{L-1} c_0^{L-1}{L-1 \choose p}  G(h, i, p) + C_i  \sum_{p=1}^{h-i} \left[\frac{c_0^p}{(1-c_0)^{p+1}} - \sum_{L=0}^{\max^{b - 1}_{h - i} - 1} c_0^{L}{L \choose p}  \right]G(h, i, p)\\
			&= C_i \sum_{L=b}^{h-i} \sum_{p=1}^{L-1} c_0^{L-1}{L-1 \choose p}  G(h, i, p) + C_i  \sum_{p=1}^{h-i} \left[\frac{c_0^p}{(1-c_0)^{p+1}} - \sum_{L=1}^{\max^{b - 1}_{h - i} - 1} {L \choose p} c_0^L \right]G(h, i, p)  % Since j=0 term is zero.
		\end{align}
		Having removed any infinities, this equation is now computationally feasible. It can be used in further calculations like the probability a player kills their opponent before they are killed.









	% If we define,
	% \begin{align}
	% 	A_{j, k} \equiv \sum_{i=0}^{\min(\lfloor j / m \rfloor, k)} a_{i, j-mi}^k,
	% \end{align}
	% then,
	% \begin{equation}
	% 	\sum_{j=0}^\infty \sum_{i=0}^k a_{i, j}^k y^{mi +j} = \sum_{j=0}^\infty A_{j, k}y^j.
	% \end{equation}
	% and so,
	% \begin{align}
	% \sum_{k=1}^\infty \left[ \sum_{j=0}^\infty \sum_{i=0}^k a_{i, j}^k y^{mi +j}\right]y^k x^k &= \sum_{k=1}^\infty \left[\sum_{j=0}^\infty A_{j, k}y^j \right] y^kx^k\\
	% &= \sum_{k=1}^\infty \sum_{j=0}^\infty A_{j, k} y^{k+j}x^k\\
	% &= \sum_{k=1}^\infty \sum_{n=k}^\infty A_{n-k, k} y^{n}x^k\\
	% \end{align}
	% Therefore,
	% \begin{align}
	% \frac{1}{1-B(y)x} &= 1 + \sum_{k=1}^\infty \sum_{n=k}^\infty A_{n-k, k} y^{n}x^k\\
	% \implies\frac{x}{1-B(y)x} &= x + \sum_{k=1}^\infty \sum_{n=k}^\infty A_{n-k, k} y^{n}x^{k+1}\\
	% &= x + \sum_{k=2}^\infty \sum_{n=k-1}^\infty A_{n-k+1, k-1} y^{n}x^k\\
	% \end{align}
	% Now,
	% \begin{align}
	% T(y)\frac{x}{1-B(y)x} &= T(y)x + T(y)\sum_{k=2}^\infty \sum_{n=k-1}^\infty A_{n-k+1, k-1} y^{n}x^k\\
	% &= T(y)x + c_+ \sum_{h=1}^m (m - h + 1)y^h\sum_{k=2}^\infty \sum_{n=k-1}^\infty A_{n-k+1, k-1} y^{n}x^k\\
	% &= T(y)x + c_+ \sum_{k=2}^\infty \sum_{n=k-1}^\infty \left[\sum_{h=1}^m (m - h + 1)A_{n-k+1, k-1} y^{n+h} \right]x^k\\
	% \end{align}
	% Define,
	% \begin{align}
	% 	\gamma_{n, k}^h \equiv (m - h + 1)A_{n-k+1, k-1}
	% \end{align}
	% Now,
	% \begin{align}
	% 	T(y)\frac{x}{1-B(y)x} &= T(y)x + c_+ \sum_{k=2}^\infty \left[\sum_{n=k-1}^\infty \sum_{h=1}^m \gamma_{n, k}^h y^{h+n} \right]x^k\\
	% \end{align}
	% To handle this, we make use of the identity [visual proof is also omitted]:
	% \begin{align}
	% 	\sum_{n=b}^\infty \sum_{h=1}^m a_{n, h} y^{n+h} = \sum_{n=b+1}^\infty\left( \sum_{h=1}^{\min(n-b, m)}a_{n-h, h} \right) y^n
	% \end{align}
	% So now,
	% \begin{align}
	% 	g(x, y) &= T(y)x + c_+ \sum_{k=2}^\infty \left[\sum_{n=k}^\infty\left( \sum_{h=1}^{\min(n-k+1, m)}\gamma_{n-h, k}^h \right) y^n \right]x^k\\
	% 	&= T(y)x + \sum_{k=2}^\infty \sum_{n=k}^\infty\left( \sum_{h=1}^{\min(n-k+1, m)}c_+\gamma_{n-h, k}^h \right) y^n x^k\\
	% \end{align}
	% With a change of variables we get,
	% \begin{align}
	% 	\sum_{L=1}^\infty \sum_{h=1}^\infty P_{h, L} y^h x^L &= T(y)x + \sum_{L=2}^\infty \sum_{h=L}^\infty\left( \sum_{i=1}^{\min(h-L+1, m)}c_+\gamma_{h-i, L}^i \right) y^h x^L\\
	% 	\sum_{h=1}^\infty P_{h, 1} y^h x + \sum_{L=2}^\infty \sum_{h=1}^\infty P_{h, L} y^h x^L &= T(y)x + \sum_{L=2}^\infty \sum_{h=L}^\infty\left( \sum_{i=1}^{\min(h-L+1, m)}c_+\gamma_{h-i, L}^i \right) y^h x^L\\
	% 	% \implies P_{h, L} &= c_+ \sum_{j=1}^{\min(h - L + 1, m)} \gamma_{h-j, L}^j\\
	% \end{align}
	% Since the $x$ terms are trivially a boundary condition, let's focus on the double sums:
	% \begin{align}
	% 	\sum_{L=2}^\infty \sum_{h=1}^\infty P_{h, L} y^h x^L &= \sum_{L=2}^\infty \sum_{h=L}^\infty D_{h, L} y^h x^L\\
	% 	% \implies P_{h, L} &= c_+ \sum_{j=1}^{\min(h - L + 1, m)} \gamma_{h-j, L}^j\\
	% \end{align}
	\newpage
	\section{Summary}
		First, given a player's max hit of $m$ and an opponent's initial health of $h$, we define:
		\begin{align}
			c_+ &= \frac{a}{m+1} \\
			c_* &= mc_+ \\
			c_0 &= 1 - c_* \\
			C_i &= c_+ (m - i + 1)
		\end{align}
		where $c_0$ is the probability of doing zero damage, and $c_+$ is the probability of doing any positive amount of damage.
		Then, the probability of killing in $L$ turns is given by the recursive equation:
		\begin{align}
			P_{h, L} &= c_0P_{h, L - 1} + c_+ \sum_{i=\max{h - m, 1}}^{h-1} P_{i, L-1},\,\,\, L \ge 2, h \ge 1
		\end{align}
		The boundary conditions are given by:
		\begin{align}
			P_{h, 1} &= c_+ \max(m - h + 1, 0)\\
			P_{1, L} &= c_* c_0^{L-1}\\
			P_{1, 1} &= c_*
		\end{align}
		This has the following solution:
		\begin{equation}
			P_{h, L} = P_{h, 1}c_0^{L-1}  + \sum_{i=1}^{\min(h, m)} H_{h, i}^L,
		\end{equation}
		where
		\begin{align}
			H_{h, i}^L &\equiv c_+c_0^{L-1} (m - i + 1)\sum_{p=1}^{\min_{h-i}^{L-1}} {L-1 \choose p}  G(h, i, p) \\
			G(h, i, p) &\equiv \left(\frac{c_+}{c_0}\right)^p  \sum_{l=0}^{\min_{\lfloor (h-i-p) / m \rfloor}^{p}} (-1)^l  {p \choose l} {h-i-ml-1\choose h-i-ml-p}.
		\end{align}
		The probability of winning a fight with an opponent, and drawing is given by:
		\begin{align}
			P_\text{win} &= \sum_{L=1}^\infty \sum_{l=L+1}^\infty P_{h_\text{player}, l}^{m_\text{opponent}} P_{h_\text{opponent}, L}^{m_\text{player}}\\
			P_\text{draw} &= \sum_{L=1}^\infty P_{h_\text{player}, L}^{m_\text{opponent}} P_{h_\text{opponent}, L}^{m_\text{player}},
		\end{align}
		where $P_\text{win} + P_\text{lose} + P_\text{draw} = 1$ and swapping opponent values for player values turns $P_\text{win}$ into $P_\text{lose}$. The sum $\sum_{L=a}^\infty P_{h, L}$ can also be expressed in terms of a finite sum.

	\section{Comments}
		It would be useful to compute the time complexity required for evaluation. There is an infinite sum in both $P_\text{win}$ and $P_\text{lose}$ that must simply be truncated during evaluation. It would be nice to either find analytic solutions or determine appropriate cutoffs.
	
